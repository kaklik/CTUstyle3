After initial processing of storm activity records with the described instruments, it became apparent that some observations are difficult to explain solely with the data obtained so far. For instance, the finding of the extensive spatial and temporal scale of lightning discharges described in section \ref[triggered_recording] raised questions about the distribution of charges within storm clouds that the lightning connects. Similarly, ambiguities arose regarding ionizing radiation. The uncertainty lies in whether ground measurements missed ionizing radiation because the detectors were too far from the radiation source or because the radiation was emitted in a direction undetectable by ground-based detectors situated directly under the storm cell. To resolve these issues, spatial data on charge distribution and the radiation field are necessary, prompting the proposal of additional supportive experiments, which allow direct in-situ measurements in the atmosphere.

\sec[balloon_flights] Stratospheric Balloon Flights

Stratospheric balloon flights are conducted for several reasons related to the research of electromagnetic phenomena in the atmosphere associated with storm activity. They allow for the testing of instruments under various adverse conditions, enable the testing of data transmission methods over relatively long distances and in unfavorable weather, and provide a platform to lift additional instruments in the atmosphere. For example, they offer insights into the behavior of ionizing radiation in the upper layers of the atmosphere, where particles of primary cosmic radiation with sufficient energy interact and generate showers of secondary cosmic radiation. This phenomenon causes the situation that with increasing depth of the atmosphere, the intensity of primary radiation decreases whereas the secondary component intensity increases. 
At an altitude of about 20 km, the intensity of secondary cosmic radiation reaches its maximum, called the Pfotzer-Regener maximum \cite[REGENER1935]. The maximum varies with geomagnetic vertical cutoff rigidity and with solar cycle and it is generally located at 15-27 km above sea level \cite[Bazilevskaya1998]. 
Due to that physical phenomenon, stratospheric balloons are a useful tool for the investigation of cosmic radiation at high altitudes (around and above the Regener-Pfotzer maximum region).

Since the composition of the radiation field varies with altitude \cite[10.1093/oxfordjournals.rpd.a081837] various radiation detector types should be used to measure the vertical profile of cosmic radiation in the atmosphere. 

However, the radiation-measuring instruments need to be supplemented by other sensors measuring temperature, pressure, humidity, altitude, acceleration, etc.  All these sensors should be continuously monitored during the launch and the flight of the balloon to verify their proper function and their values have to be recorded for further processing of all obtained data. 

Therefore I proposed the development of the TF-ATMON\urlnote{https://www.thunderfly.cz/tf-atmon.html} system, based on the use of existing tools of the open-source PX4 autopilot and supplemented with the new TFSIK01 telemetry transmitter. That enables on-ground monitoring of sensors and recording of data during the flight and at the same time real-time tracking of the balloon. This methodology was firstly used in full scale on stratospheric balloon flight FIK-6 \ref[FIK-6_setup]  and then improved in FIK-9 flight. 

I have participated in the launch of  multiple stratospheric balloons, the overview of flights is summarized in Table 1 in appendix \ref[FIK_flights]. At the beginning (until the FIK6 flight) the construction of balloon gondolas was designed for specific instruments used for a particular flight. Balloon avionics was therefore built around the chosen detectors from scratch.

This concept led to a situation when every new flight meant a significant amount of work although many components were recycled every year and used for the next one. The reason was the need to adapt the avionics to the updated version of the detectors.

Due to a relatively high value of payload and measurement relied only on board data recording it was necessary to ensure the recovery of the gondola after every flight. Therefore the main criteria for the construction of the balloon experiment were as follows:

\begitems
* Reliable transmission of information about the geographical location of the gondola 
* Good resistance to impact
* Ensuring the function in temperatures far below zero
\enditems

Despite the generally successful nature of all fights and the fact that the gondolas were always found, different technologies were tried to eliminate the partial shortcomings that emerged during the previous balloon flights. For example, the initially used GSM-based telemetry system was replaced by a significantly more robust IoT LoRa transmitter, making it possible to transmit the data necessary for tracing the gondola directly to TheThingsNetwork\urlnote{https://www.thethingsnetwork.org/}. This step enables a high reliability of finding the gondola and at the same time a basic recording of the experiment data transmitted to the ground. For the case of more advanced detectors, it was also necessary that the 3D flight trajectory including the orientation be recorded synchronously with the supplementary quantities. That is also ensured with the incorporation of the TF-ATMON system, which relies on PX4 autopilot firmware running on Pixhawk hardware equipped with IMU sensors. As can be seen from Table 1 in appendix \ref[FIK_flights], flights conducted since FIK-5 implemented avionics using UAV technology. It is realized by the use of the Pixhawk-based autopilot hardware with PX4 firmware. Telemetry was implemented by a very reliable combination of the LoRa modem and SiK modem. 

Apart from technologies used directly in balloon  gondolas, many supplementary tools on the ground have undergone intensive development. For example, to find the balloon it was necessary to have an accurate real-time map of its position together with a prediction of the next stage of the flight and the location of touch down. For that case of the flights since the FIK-5 used the HabHub \urlnote{https://tracker.habhub.org/} (lately the SondeHub\urlnote{https://amateur.sondehub.org/}) was used to track and predict the movement of stratospheric balloons in real-time. In fact, in the beginning, the HabHub was used to manually run the prediction of flight, but the real-time calculation of trajectory predictions introduced in FIK-3 based on live telemetry data significantly decreased the time required for gondola recovery. 

\secc The design of universal avionics

Based on the described experiences the concept of a universal avionics called TF-ATMON emerged. That technology makes it possible to connect different types of payloads and carry out various atmospheric measurements. Furthermore, it provides basic services such as power supply, time, position, and 3D orientation information to the payload instruments. That data is recorded in parallel with quantities that affect many types of measurements, including temperature, pressure, humidity, magnetic field, and acceleration. At the same time flight computer provides the services for payload to record data from experiments in a common log file. 
Therefore the TF-ATMON onboard stratospheric balloons have become an extremely useful tool for testing instruments like cosmic radiation detectors and dosimeters at high altitudes.

The schematic diagram of the avionics hardware is summarized in the figure \ref[avionics_schematics].

\medskip
\clabel[avionics_schematics]{FIK flight avionics}
\picw=15cm \cinspic ./img/avionics_block_schematics.pdf
\caption/f The schematic diagram of the new avionics concept used in FIK flights. 
\medskip

The concept, where the balloon-specific avionics are almost completely separated from the experiment set of detectors, simplified the realization of the next balloon flights. It reduced the complexity of connecting different types of detectors and at the same time, it improved the integrity of supplementary data measurements. Overall, the new features can be summarized as follows:

\begitems
* Easy implementation of different payloads
* Redundant telemetry links
* Gondola orientation and spatial position tracking and logging
* Reliable IMU sensor processing and calibration
* Possible use of relatively high-power consumption payloads
* Pre-flight continuous charging as an option
* Power monitoring and uptime calculation relevant to the actual temperature and available energy level
* Real-time pre-flight payload diagnostic
\enditems

The detailed documentation of used electronics blocks can be found under the following names TFGPS01, TFSIK01, TFHT01, TFLORA01. It should be noted that the TFSIK modem has been designed after the long use of the MLAB electronic module ISM01, which has been its predecessor. In both cases, the SiK firmware used on the balloon has been altered to one-way (downlink) communication (therefore the modem on the balloon is not able to receive) for safety reasons. 

\secc Instrument payload

In the case of FIK-5 and FIK-6 flights that served as the test flights for the novel approach using TF-ATMON technology the payload was not fully relied on yet. All the detectors thus had their SD cards for data recording and some even had their power supply, therefore the payload weight was higher than theoretically required in case of full use of TF-ATMON and some lift was wasted. This situation originated in the conservative flight plan, which required a successful log and function of payload even in the failure of the new method.

\secc Atmospheric phenomena Detectors 

The payload for example FIK-6 flight contained TF-ATMON and three different types of ionizing radiation detectors already mentioned in \ref[instrumentation] section: SPACEDOS with silicon PIN diode sensor, AIRDOS-C with scintillation crystal and silicon photomultiplier and a G-M tube. The total payload mass was 2 kg.

SPACEDOS is a lightweight dosimeter intended for space applications and measurements onboard spacecraft. The detector has been described in \cite[10.1093/rpd/ncac106]. AIRDOS-C is a scintillation detector with a small crystal. The detector has been described in \cite[10.1093/rpd/ncac105]. The G-M tube STS-5 was used in the Geiger–Müller counter. This detector is capable of registering the flux only and is included because it has high detection volume, which allows good resolution in flux. All detectors, together with other sensors and the TF-ATMON system were put inside a polystyrene box. 

\medskip
\clabel[FIK-6_setup]{FIK-6 flight balloon setup}
\picw=8cm \cinspic ./img/FIK-6_experiment_setup.png
\caption/f FIK-6 experiment setup using the Hwoyee Weather Balloon 1600. 
\medskip

\secc Results from the balloon flights 

The flight FIK-6 took place on December 18th, 2020 and lasted 1 hour and 40 minutes. The balloon was launched from Příbram airport (LKPM) which is located around latitude 50$^{\circ}$ N. The balloon flight path continued in the eastern direction for about 80km. 

The system TF-ATMON recorded temperature, air pressure, humidity, and radiation characteristics as histograms of deposited energy of radiation events from all three radiation sensors in the gondola, see Figure \ref[FIK-6_RAW_data]. The barometric altitude was calculated using the International Standard Atmosphere Model 1976 \cite[us_standard_atmosphere_1976]. 

\medskip
\clabel[FIK-6_RAW_data]{FIK-6 flight recorded data}
\picw=15cm \cinspic ./img/FIK-6_RAW_data.png
\caption/f Raw data measured during the flight. From top to bottom: temperature near scintillation crystal, air pressure inside the box of crystal, temperature inside the gondola, relative humidity inside the gondola, counts of radiation events per 10 seconds counted by scintillator, silicon detector, and  G-M counter, barometric altitude, and altitude from GNSS.
\medskip

\medskip
\clabel[FIK-6_telemetry]{Acceleration and Angular Speed balloon telemetry}
\picw=15cm \cinspic ./img/FIK-6_metadada.png
\caption/f Acceleration and angular speed in axis perpendicular to the ground combined with ionizing radiation flux measured by the silicon PIN diode detector.
\medskip

Figure \ref[FIK-6_telemetry] demonstrates the importance of telemetry data from the TF-ATMON system during measurement processing. The graphs show an increase in the response of the silicon ionizing radiation detector at the times of take-off, burst, and landing when there was a rapid increase in mechanical stress. The effect is caused by the microphone effect of the silicon detector circuit. At the same time, it can be seen there are considerable vibrations, rapid changes in acceleration, and gondola rotation during the balloon descent. The increase in relative humidity is also observable in fig \ref[FIK-6_RAW_data]. Then the water wapor can freeze on the instruments during some parts of the flight. All of them may affect the measurement of atmospheric quantities and for some types of instruments, they have to be compensated.

The graphs show that the maximum reached altitude was approximately 33 km above sea level. During the flight, the balloon passed the Regener-Pfotzer maximum twice. The rescue team followed the balloon along the whole flight trajectory. The precision of tracking allowed some participants of the rescue team to see the gondola touchdown visually. Therefore the gondola was successfully rescued within a few minutes after touchdown (Figure \ref[FIK-6_rescue_team]).

\medskip
\clabel[FIK-6_rescue_team]{Gondola recovery, based on trajectory prediction}
\picw=8cm \cinspic ./img/FIK-6_rescue_team.png
\caption/f A screenshot from HabHub tracker. The CRREAT measurement vehicles, serving as stratospheric balloon chase cars, are equipped with telemetry receivers at the landing site. They are prepared for immediate gondola recovery, thanks to precise touchdown location predictions.
\medskip

By processing the data a graph of detected altitude Fig. \ref[R-P_maximum] was obtained. It shows that the measured altitude of the Regener-Pfotzer maximum was for all detector types around 19 km above sea level. 
As the measured data show in Fig \ref[R-P_maximum], there is a very noticeable difference between a number of data points measured during the flight upwards and during the descent. This difference is mainly caused by the different values of vertical speed. In future balloon flights, this problem is planned to be overcome by a controlled descent, during which the rate of descent can be decreased in some phases of the flight so that it can be more comparable with the speed of the ascent.

\medskip
\clabel[R-P_maximum]{Estimated Regener-Pfotzer maxima}
\picw=15cm \cinspic ./img/FIK-6_R-P_maximum.png
\caption/f Log-norm fit of ionizing radiation measured data with calculated Regener-Pfotzer maxima for different detectors (G-M tube, NaI(Tl) scintillator, and silicon PIN diode). Measured data are depicted with gray color as mean from 5 minutes of measurement with 2$\sigma$ standard errors.
\medskip



\secc Outcome of stratospheric balloon flights

The FIK flights were  used to test different types of ionizing radiation detectors with the use of the Regener-Pfotzer maximum phenomenon. 

The telemetric system \iid TF-ATMON has been verified. It enables data recording, pre-flight instruments check, and their monitoring during flight. Thanks to the availability of different communication interfaces  to the avionics, the use of various payload detectors is simplified. The technology at the same time improves the possibilities of  the fast localization of a balloon gondola after its landing. It is therefore possible to carry out even experiments requiring a short time to recover the gondola after the flight. 

As can be seen from Table 1 in appendix \ref[FIK_flights], the last unsolved problem with the stratospheric balloon flights is the control of descent. Therefore, in future balloon flights and thunderstorm exploration, there exists a requirement to use the autopilot contained in TF-ATMON to control descent trajectory.

The descent should be controlled by using an airframe carrying a payload. Thus it would be possible to choose the landing site and reduce the possible risk of creating dangerous situations and at the same time with a possibility to control the descent rate and trajectory. 
Another unresolved challenge for use of the balloons for thunderstorm research is the difficulty of launching them in strong winds, which are nearly always present near storms, especially at times when the measurements are most critical. See the Figure \ref[balloon_wind_takeoff] for that situation.
For these reasons, it became necessary to shift to other carrier platforms to be used for practical storm activity measurements. The logical alternative for this step is to employ unmanned aircraft system (\glref{UAS}).

\medskip
\clabel[balloon_wind_takeoff]{Takeoff of the stratospheric balloon during high wind conditions}
\picw=15cm \cinspic ./img/balloon_wind_takeoff.jpg
\caption/f The experimental stratospheric balloon takeoff (FIK-2) under increased wind conditions. The takeoff was partially successful; however, the gondola impacted me and then the ground, causing damage to the internal power supply and telemetry antenna.
\medskip

\sec[autogyro_thunderstorm] Autogyro Thunderstorm Flights

Despite widespread efforts which could be found in the literature \cite[https://doi.org/10.1029/2006JD007242,TELEXTheThunderstormElectrificationandLightningExperiment,  ScientificInsightsfromFourGenerationsofLagrangianSmartBalloonsinAtmosphericResearch, ABalloon_precipitation] to deploy measurement balloons into thunderstorms with various sensors, these attempts were often unsafe, and the results are sporadic because the balloon flight path is uncontrolled or control is very limited \urlnote{https://windbornesystems.com/}. Therefore that technique seems to be outdated in comparison with the current \glref{UAV} technology. This fact led to the exploration of using \glref{UAV}s instead. Moreover, due to the spatial extent of lightning structures (see section \ref[results] for details), it's required to experiment with multiple coordinated locations simultaneously, which is practically unattainable with uncontrolled balloons.  Therefore better understanding of the phenomena can very likely be achieved by using specially equipped \glref{UAS}, which enables measurement directly in or near the thunderstorms. 

\glref{UAS} offer additional advantages, such as the ability to measure in specific atmospheric layers, where it is required to be actively held, which is useful for determining key parameters that influence cloud electrification, like radon and aerosol concentrations on the entry of lifting condensation level (\glref{LCL}) for example. The obvious requirement is to use a new unmanned airframe using TF-ATMON described in section \ref[balloon_flights] and employ the integrated flight controller to take control over the measurement procedure in the atmosphere.

\medskip
\clabel[TF-G2_fly_clouds]{TF-G2 during flight}
\picw=15cm \cinspic ./img/TF-G2_fly_clouds.jpg
\caption/f TF-G2 autogyro under the clouds, during one of many test flights. 
\medskip

The described idea resulted in the attempt to use an unmanned autogyro to carry measuring instruments to the vicinity of thunderstorms shown in Figure \ref[TF-G2_fly_clouds]. The main goal of this experiment was to measure and locate the ionizing radiation presence together with the electric field and resolve the uncertainty of the source and direction of ionizing radiation produced in thunderstorms. 

For the investigation of thunderstorm activity, an autogyro presents several key advantages over other types of UAVs such as multicopters or fixed-wing aircraft, primarily due to its unique aerodynamic properties and operational capabilities. Autogyros, or gyroplanes, leverage the autorotation of their main rotor to stay airborne, which allows for safe flight at low speeds and in turbulent atmospheric conditions often encountered near thunderstorms.

Compared to multicopters, autogyros are more efficient in terms of energy consumption for longer flight tracks, as they do not require power and control to spin multiple rotors for lift, relying instead on forward motion or wind. This efficiency is advantageous for conducting high-altitude atmospheric research missions that demand sustained presence in targeted areas to gather comprehensive data.

Moreover, unlike fixed-wing aircraft that require higher speeds and runways for safety, autogyros can perform take-offs and landings in confined spaces, making them ideal for deployment in varied terrains and closer proximity to storm activities. Their inherent stability, even in gusty conditions, allows for more reliable data collection of meteorological parameters.

\secc Autogyro airframe

There was not a suitable unmanned autogyro to perform atmospheric measurement in stormy conditions commercially available. The situation resulted in the design of TF-G2 autogyro which I developed in cooperation with the ThunderFly team. This step represents a significant advancement in the design and application of UAV technology for atmospheric research, specifically targeting the challenging conditions presented by thunderstorms. 
The TF-G2 autogyro's airframe utilizes a lightweight yet robust and high-impact strength 3D printed material for the airframe and unique rotor design, which enables rotor blade shape modification. This technological decision gives the ability to easily replace parts, which is useful to quickly resolve damages as flights near thunderclouds often result in mishaps. 
Internal components are protected from the weather by a waterproof fabric cover, which enables easy service access and supports equipment variability with its shape adaptability. Furthermore, the airframe's design allows easy access to onboard instruments and maintenance. This capability is achieved by rack-style mount options, where the payload could be easily fixed to a grid of screw holes which can be seen in figure \ref[TF-G2_hangar].

\medskip
\clabel[TF-G2_hangar]{TF-G2 without rain cover}
\picw=15cm \cinspic ./img/TF-G2_hangar.png
\caption/f A development version of TF-G2 autogyro, waterproof fabric cover removed. The 3D-printed parts could be seen according to the grid of screw holes used for mounting of the experimental equipment. 
\medskip

The avionics design is largely identical to the avionics used in already described stratospheric balloon flights, with differences primarily in the power supply, because the autogyro includes components for propulsion, such as an electric motor and its electronic speed controller (\glref{ESC}). To ensure robust pre-flight control, the ESC provides status information, including energy drawn from batteries via the UAVCAN bus. Another distinction is the use of special sensors, like an airspeed sensor, half-duplex telemetry system, and specific sensors used directly for atmospheric measurements.

\secc Airspeed sensing

In the case of autogyros, similar to airplanes, it is required to measure the indicated airspeed (\glref{IAS}) to safely perform takeoffs and in-flight maneuvers. Even though autogyros cannot be stalled in the traditional sense because their lift is generated by the rotor's rotational speed rather than directly by airspeed, they exhibit unique behavior. At insufficient forward airspeed, a gyrocopter begins to settle until it potentially impacts the ground — a scenario that can be considered a survivable landing. Conversely, when the airspeed is excessively high, the rotor can easily exceed its critical \glref{RPM}, leading to potential rotor disintegration. To mitigate these risks, the TF-G2 autogyro is equipped with an \glref{IAS}  sensor to accurately monitor and control airspeed, thus ensuring operational safety under varying flight conditions.
Initially, a conventional Pitot tube was used for that purpose, but it proved very sensitive to clogging by snow, ice, or mud after complicated landings, necessitating disassembly and thorough cleaning. Thus, I proposed a solution named TFSLOT01, utilizing the Venturi effect.  To describe the application of Bernoulli’s principle for measuring airspeed, we define the following variables:

\begitems
* $\Delta p$ -- Measured pressure difference
* $\rho$ -- Air density
* $v_\infty$ -- Free air velocity e.g. measured airspeed
* $v$ -- The velocity of air running through the sensor
* $A_D$ -- Cross-section area at the position of the outer pressure port
* $A_d$ -- Cross-section area at the position of internal pressure port
\enditems

In the context of small velocities, where the TFSLOT01 sensor operates, the air velocity and pressure must satisfy Bernoulli’s principle:

$$ {1 \over 2}\rho v_\infty^{2} + p_\infty = {1 \over 2}\rho v^{2} + p \eqmark[Bernoulli]$$

Then the velocities are in relation to cross sections in the plane of pressure measurement ports

$$ {v \over v_\infty} = {A_D \over A_d} \eqmark[cross_section_areas]$$


Therefore equations for pressure difference and airspeed velocity could be derived, which correspond to the \glref{IAS}.

$$
\Delta p = {1 \over 2}\rho{v_\infty}^{2}\left[\left({v \over v_\infty}\right)^2 - 1\right] \Rightarrow v_\infty = \sqrt{{2\Delta p \over \rho\left[\left({A_D \over A_d}\right)^2 - 1\right]}}
\eqmark[air_velocity]
$$

There should be noted that the sensor utilizing the described principle is theoretically more sensitive (e.g. has a higher measurable pressure difference at the same airspeed) than the Pitot-static tube (at the same air density) in the geometric configuration where the following equation is valid

$$ {A_D \over A_d} > \sqrt{2} \eqmark[IAS_sensitivity]$$

The advantage of the higher sensitivity results in increased drag, but it should be negligible at low airspeeds. Therefore the best use case of the sensor is the integration of the sensing device into the relatively massive fuselage or other suitable airframe structure. In the case of TF-G2 autogyro, it is resolved by integration of the TFSLOT01 sensor directly to the rotor head, just below the rotor hub as could be seen in Figure \ref[TF-G2_hangar]. Figure \ref[TFSLOT01] shows details of the TFSLOT01 sensor. 

\medskip
\clabel[TFSLOT01]{TFSLOT01 airspeed sensor design}
\picw=15cm \cinspic ./img/TFSLOT01.jpg
\caption/f The photograph showcases a close-up of TFSLOT01, where the front part features the entry aperture $A_D$. The narrow aperture $A_d$ is slightly deeper into the structure. In the rear of the photo is visible the I2C cable, used to connect  the sensor to the flight controller. 
\medskip

\secc Telemetry system


The telemetry system used in the \iid TF-G2 autogyro is significantly improved but still based on the technology previously tested on the stratospheric balloon's telemetry link. Unlike the one-way radio communication utilized in the balloon experiments, the TF-G2 employs a bidirectional half-duplex, TDM radio system. The modem still operates at a frequency of 433MHz, but with an improved RF front-end to ensure a more reliable communication link due to the demanding environmental conditions of the TF-G2 flight. The modem firmware is again based on modified SiK firmware, which incorporates integration to the new hardware capable of supporting antenna diversity, specifically a 2x2 MIMO configuration. This arrangement allows for the connection of two external antennas, each of which can be utilized for both transmitting and receiving signals. The optimal antenna is selected based on the RSSI metric on the received packet preamble before each transmission of the next packet. This dual-antenna setup significantly improves signal resilience against interference from multipath fading and is also robust against mechanical damage to one of the antennas—a critical improvement over the previous single-antenna modems used on balloon flights, where the damage actually happens (see Table 1 in appendix \ref[FIK_flights]). It is also quite a common failure mode of the antenna on UAVs, caused by prolonged exposure to vibrations during flights.

\medskip
\clabel[TFSIK01]{TFSIK dual-antenna telemetry modem}
\picw=15cm \cinspic ./img/TFSIK01.jpg
\caption/f TFSIK PCB without electromagnetic shielding and 3D printed housing. The RF input and output with impedance matching network, RF switch, band-pass filter, and LNA are on the left of the photo. 
\medskip


Furthermore, the improved system benefits from the versatility in antenna configurations at the ground station side, where a combination of the two antennas (directional and omnidirectional antenna) ensures optimal communication over varying distances. This dual-antenna approach at the ground station seamlessly switches between short-range and long-range communication modes without user intervention, offering a reliable link irrespective of the TF-G2's distance from the ground station.

\secc Carried sensors

The key innovation in the TF-G2's design is the tight integration of specialized sensors for measuring electric fields and ionizing radiation. These sensors are mounted in a way to minimizes interference from the airframe by maximizing the benefits of the autogyro construction. The placement of an electric field mill is directly under the rotor, exploiting the UAV rotor's characteristic of being unpowered, thus avoiding significant electromagnetic interaction with the measured values. See the figure \ref[E_mill_rotor]. The only additional requirement on the airframe is that the rotary disk of EFM is electrically connected to the measuring electronics. That is achieved through a conductive connection of the rotor's bearing.
For the detection of ionizing radiation, the UAV has been equipped with a semiconductor detector, AIRDOS03 (UAVDOS), previously tested during balloon flights. It is situated in the UAV's fuselage under a waterproof fabric cover and connected to the TF-ATMON system via serial link similarly as described in section \ref[balloon_flights]. The avionics system manages sensor readout, synchronizing data collection with the TF-G2's flight path and with data from other sensors (EFM, humidity, etc.), see Fig. \ref[E_mill_data] for an example of that data. That onboard data processing together with the telemetry link enables real-time preliminary analysis, aiding in the assessment of collected data during the flight.

\medskip
\clabel[E_mill_rotor]{EFM mounted on autogyro's rotor head}
\picw=15cm \cinspic ./img/E_mill_rotor.png
\caption/f Close up view of electric field mill (THUNDERMILL01) mounted into the autogyro rotor head. The sensing electrodes are below the rotating grounded disk. Part of one sensing electrode is visible in the left side of the E-mill’s rotor disk opening.  
\medskip

\medskip
\clabel[E_mill_data]{Example of data measured by EFM mounted on autogyro}
\picw=15cm \cinspic ./img/E_mill_data.png
\caption/f Electric field measurement data along with environmental humidity and temperature, aligned with flight parameters during a flight in fair weather. 
\medskip


\secc Takeoff technique 

Since the unmanned autogyro TF-G2 (like other autogyros) requires the rotor to be spun up to a flight RPM value to allow its operation during the flight, it is necessary to inject rotational energy into the rotor before takeoff (in the case of TF-G2 the required energy range is something between 100 to 200 J, depending on the actual weight and rotor configuration). One of the simplest ways to achieve this spin-up is by ensuring airflow through the rotor, which is firstly pre-spun to a minimum RPM by directly applying a rotation force to the rotor. Subsequently, these rotations increase to flight speed through forward motion. The required minimum forward speed needed to increase RPM to flight value is in the range of a few meters per second. In the case of those conditions, the autogyro becomes airborne within a few tens of meters, after that the internal autopilot is able to automatically climb to a stable flight.
To ensure this procedure is feasible during approaching thunderstorms, it was necessary to develop a range of supportive devices, such as a launch platform and visualization of the gyrocopter's status on the car roof. The takeoff from the car roof platform is an advantage for safety and time-saving.
The resulting design of the takeoff platform is visible in Figure \ref[autogyro_takeoff_platform], where the autogyro is secured in an inclined launch position, allowing to increase in the airflow through the rotor. In this position, the gyrocopter's rotor is locked by a mechanism contained in the platform until takeoff. At the beginning of this assisted takeoff, the platform unlocks the rotor, which is then spun to minimum speed by weights connected to the rotor by pulleys. At this minimum RPM speed, aerodynamic forces enable further rotor acceleration due to the incoming airflow.
However, since the launch platform is mounted on the roof of the vehicle, the crew in the vehicle (pilot and driver) cannot directly see the procedure. Therefore, it is mandatory for the vehicle's crew to receive adequate and high-quality data to semi-automatically carry out the launch procedure. The TF-G2's autopilot is programmed to automatically transition between the different phases of takeoff once predefined criteria are met. Therefore the crew only needs to monitor key decision variables such as IAS and rotor RPM, to be able to respond adequately to a non-standard situation. These values are therefore required to be displayed with low latency and in a format that enables the car driver to determine whether to speed up, slow down, or maintain the current speed. This is facilitated by a set of displays, as seen in Figure \ref[takeoff_display], which have been added to the measurement vehicle.
Additionally, the overall status is visible through an analog camera, which provides both the driver and the operator a view of the autogyro, mounted in the takeoff platform. However, this display is limited by the camera's frame rate and the resolution of the analog camera (a digital camera cannot be used due to typically high latency in the image encoder and decoder). The driver and operator also have access to auditory and visual signals indicating the platform and takeoff status via a shared handheld controller, which can be used to abort the takeoff procedure in the event of an unforeseen issue.

\medskip
\clabel[takeoff_display]{UAV control panel}
\picw=15cm \cinspic ./img/takeoff_display.png
\caption/f Measuring CAR2 was equipped with special instruments required to operate the unmanned autogyro TF-G2.  In the center is an autogyro status display with rotor RPM and IAS indicators, the bottom is video from the roof camera and on the right, there is a UAV operator’s laptop.
\medskip

The takeoff platform device is removably integrated into the assembly of magnetic VLF receiving antennas, as seen in Figure \ref[autogyro_takeoff_platform]. This platform is connected to the ground control station mounted inside the trunk of the car and is controlled by the autogyro's autopilot states via previously described TFSIK01 telemetry modems. The platform's firmware responds to commands sent by the TF-G2 using the commands on the MAVLink 2 protocol.
The mentioned handheld controller for the driver and operator is directly connected to the microcontroller managing the platform, allowing the platform's internal state to be overridden despite commands from the TF-G2. That takeoff solution is designed primarily to achieve overall safety before and during the thunderstorm measurement attempt. 

\medskip
\clabel[autogyro_takeoff_platform]{Autogyro takeoff platform}
\picw=15cm \cinspic ./img/autogyro_takeoff_platform.png
\caption/f Measuring CAR2 was equipped with a roof platform used to carry the TF-G2 autogyro to the thunderstorm site. The rotor fixing is the black arm on the left. The gray tubes are oriented vertically during actual takeoff and housing the weights used to prerotate the rotor to minimum RPMs.
\medskip

\secc Test flights

Throughout the development phase of the unmanned autogyro, extensive test flights were conducted to evaluate various modifications and enhancements to the UAV design, the firmware of the autopilot, and the ground equipment adjustments. One of these could be seen in the figure \ref[lift_drag].  Among these development verification, there were tests on detectors in certain scenarios. These are experiments usually conducted outside of the main thunderstorm session. 

\medskip
\clabel[lift_drag]{Lift and Drag measurement platform}
\picw=15cm \cinspic ./img/lift_drag.png
\caption/f Measuring CAR2 was equipped with a roof lift and drag measuring platform used to carry on-ground testing procedures. 
\medskip

Two notable supportive experiments emerged from these testing flights. The first significant experiment involved testing the UAV's flight capabilities under strong wind conditions alongside measuring floating dust concentrations (floating dust is associated with the development of storm activity and the electrification of cloudiness \cite[7735367]). For this experiment, New Year's Eve fireworks were used as a modeling source of floating dust. The measurements were significant, as expected because the instrument convincingly detected at least a two-fold increase in the concentrations of floating dust in hours following the fireworks display.
The second experiment was focused on verifying the directional homogeneity of the signal received by a Quadrifilar Helix (QFH) antenna (see figure \ref[QFH_antenna_radiation_pattern]). Leveraging the close frequency proximity of the telemetry transmission band from the UAV (433 MHz) to the observational band for lightning discharges, this experiment was conducted by having the UAV fly in circles around a measuring vehicle. The signal level was manually monitored on the receiver. This experiment was used for verification of the QFH antenna's effectiveness in maintaining homogenous sensitivity.
In addition to the mentioned experiments, numerous routine flights and experiments were conducted near storm clouds. These experiments were carried out simultaneously with ground measurements, which were the primary focus. The success of these attempts varied greatly; however, system failures often occurred, preventing the intended use of the collected data.

\sec Ground level measurements

Within the context of the CRREAT project, the term {\em ground level measurements} encompasses a broader scope, including measurements based on passive and stationary high-altitude observatories. My involvement in these stationary measurements was minimal, primarily relating to some aspects of electronics design. However, I consider it useful to mention the existence of these measurements here because uses similar, or identical equipment, which allows for the extrapolation of findings from the stationary observation’s data to active ground measurements conducted near thunderclouds with vehicles, on where my primary work was focused.

The summer storm measurement campaigns employing vehicles equipped with detection devices represent a culmination of all experiments described in previous sections. The deployment of vehicles equipped with specialized detection devices is based on exhaustive preparatory work, where each component's function and reliability are evaluated by previously mentioned experiments performed occasionally out of the main thunderstorm session e.g. in winter.  Given that the components and devices were often tested separately to prevent their failure during critical moments of storm activity. 
The average duration of monitoring a single storm's development is approximately half an hour, offering minimal opportunity for adjustments or repairs. The presence of an error on a single of three cars usually fails in a thunderstorm observation attempt. This constraint illustrates the importance of reliability, precise calibration, and software stability of the instruments used in the described research.
Throughout the campaign, the reliability of the instruments, the methods of their calibration (for example dark frame and flat-field of cameras), and the challenges posed by software errors were gradually identified as the critical barriers to conducting the research. Each of these elements significantly influenced the outcome of the measurements and, in turn, the insights that could be collected from storm activity. This reality unexpectedly highlights the intrinsic complexity of atmospheric measurements, when attempting to capture the transient nature of thunderstorm phenomena.
Recognizing these challenges, during the multiple summer storm seasons, many measuring expeditions were carried out in the Czech Republic and Slovakia with the equipment described. This allowed the measurement of electrostatic, magnetic, electromagnetic, and optical manifestations of storms, including monitoring ionizing radiation. By comparing the different obtained lightning recordings with each other, individual phases of lightning can be reconstructed.

Observations were carried out with a gradually improving strategy to get the measuring car (especially the ones equipped with radiation detectors) as close to the storm cell or under the cell as possible. The position of the storm cell was monitored by services using data from third-party networks \cite[windy,Blitzortung].

\secc Measuring Cars Infrastructure

The position of the measuring car was stationary during the period of thunderstorms. Moreover, in the case of radiation detectors, an extended static position time interval is required to capture the before and after storm activity in order to be able to record the development of radiation change including the local parameters of the radiation background. 

\medskip
\clabel[CRREAT_CAR0_diagram]{CRREAT CAR0 instrumentation diagram}
\picw=15cm \cinspic ./img/CRREAT_CAR0_diagram.png
\caption/f CRREAT CAR0 instrumentation schematic diagram. This car differs by a four-element UHF antenna array receiver and GNSS orientation tracking sensor. 
\medskip

During the measuring campaigns, the method of data recording developed as well. The first campaigns were trying to make fully automatic recordings based on observed electromagnetic signals using the described LIGHTNING01 module. Refer to section \ref[LIGHTNING01A_trigger] for details.
This system however had a lot of imperfections leading to its abandonment in the measuring car CAR0, where it was replaced by solely manual activation of recording until the trigger was implemented directly into the VLF receiver’s FPGA. See fig. \ref[CRREAT_CAR0_diagram] for overview.  
The manual activation of the recording was based on the visual perception of the operator observing the lightning activity. In contrast to CAR0, CAR1 was left with a semi-automatic method of activating the recording to compare the efficiency of both trigger methods. Car CAR1 thus had for a long time recorded triggers based on using a loop antenna and oscilloscope that generated triggers for other devices. See Fig. \ref[CRREAT_CAR1_diagram].  After the successful test of manual triggers that option was implemented in all three cars.

\medskip
\clabel[CRREAT_CAR1_diagram]{CRREAT CAR1 instrumentation diagram}
\picw=15cm \cinspic ./img/CRREAT_CAR1_diagram.png
\caption/f CRREAT CAR1 instrumentation schematic diagram. This car has the electric field mill sensor installed. 
\medskip

The trigger signal was initially distributed to the camera via the Ethernet network, a solution common to both measuring cars, but at the end of the campaign this method was abandoned and replaced with a direct trigger signal connected to the high-speed all-sky camera input because Ethernet has unknown latency on processing the trigger packet from network and also in thunderstorm conditions, there sometimes happens that trigger packet is not derived to the camera correctly.

\medskip
\clabel[CRREAT_CAR2_diagram]{CRREAT CAR2 instrumentation diagram}
\picw=15cm \cinspic ./img/CRREAT_CAR2_diagram.png
\caption/f  CRREAT CAR2 instrumentation schematic diagram. This car has special equipment for takeoff and command and control of the UAV (unmanned autogyro).  
\medskip

A significant challenge was caused by the time needed to record the data measured by individual detectors resulting from the long time interval needed to store the recording from the operational memory of devices. The dead time was gradually reduced by optimizing the settings and improving software and firmware, but it still remains the main limit preventing the gathering of data from more lightning events in case of successful expeditions.  

For high-quality data collection on storm activity, most of the described instruments require accurate information about time and position to compare the recorded data with external data sources.
GNSS receivers were mainly used for this purpose. The evolution of position tracking for the vehicle initially utilized a simple solution with a GPS receiver for coordinate logging and a manually targeted compass to determine car orientation. However, this method proved imprecise and unfeasible in the context of various operational tasks, as it was often neglected or not performed accurately.
Therefore, it was necessary to explore other methods that ideally would not depend on manual recording of the car orientation. This led to the adoption of radio navigation, enabling real-time tracking of orientation. For this purpose, the three MLAB GNSS modules GPS02 equipped with a uBlox NEO-M8P receiver capable of Real-Time Kinematic (RTK) navigation were used. That module at the same time supports moving baseline configurations, allowing each receiver to be used both as a rover and a base station. The moving baseline in the context of uBlox means the  position is measured in local NED coordinates relative to the base. Therefore a baseline and three position angles could be estimated. 

This feature of the receiver not only enabled precise positioning but was particularly beneficial for applications like UAV following or potential future implementation of landing back on the car's roof. Utilizing this for the car's platform orientation across three axes partially replaced the manually operated compass but faced challenges in signal quality due to onboard equipment interference. Additionally, such a solution is very sensitive to interruptions in the signal phase, as when the car passes under a highway bridge or driving on a road covered by trees, both can easily disrupt the signal phase and result in incorrect orientation measurements. The usability of this innovation was thus primarily hindered by the implementation in the selected receiver, which was unable to reliably indicate when the provided moving baseline coordinates were invalid.

\medskip
\clabel[RTK_moving_baseline]{RTK antennas mounted on a vehicle}
\picw=15cm \cinspic ./img/RTK_moving_baseline.png
\caption/f  This image shows the locations of the RTK (Real-Time Kinematic) antennas mounted on a vehicle for positioning using the moving baseline method.  
\medskip

It was initially assumed that the suboptimal results (even below the manufacturer's specifications) were due to poor-quality GNSS antennas. Basic patch active antennas were used, all oriented in the same direction according to the application note, to prevent phase shifts from different directions. However, changing the antenna type to a QFH antenna, which has significantly better parameters compared to the patch antenna, did not result in any noticeable improvement in the receiver's performance and ability to correctly mark invalid data.

To obtain precise system time for all onboard instruments in the measuring car, GNSS receivers were utilized. Initially, this was achieved by combining the relative PPS signal (as described in the section \ref[VLF_receiver]) with the absolute time from the internet, synchronized with the operating system time using {\em ntp-wait}. However, this method had drawbacks, it required a stable internet connection and took a long time to synchronize. Internet connectivity was provided via LTE, which was unreliable in some regions and during thunderstorms, leading to complications in initializing the measurement system and difficulty in restarting the instrumentation after power outages, such as when the car's diesel engine was restarted. 
The latest solution involved connecting an additional GNSS receiver to the onboard router, Turris Omnia, which provided basic positioning and precise time information. This router then served as the accurate time source for the onboard Ethernet network, significantly improving the reliability of the entire setup.

From the beginning of utilizing cars for atmospheric measurements, challenges were faced in how to effectively share information between drivers or operators of the measuring vehicles. Initially, with only two vehicles, the crew included an operator besides the driver to manage navigation, search for suitable observation spots, respond to the actual storm developments, and assist with instrument operation. However, this approach became impractical with the addition of a third vehicle due to the demand for a high number of personnel. The first attempt at solving this issue involved using PMR radios for inter-vehicle communication, which quickly proved ineffective for drivers and did not support the inclusion of a coordinating team member. Expanding PMR radios with chat and location sharing via standard communication apps also fell short in practicality.

Thus, HabHub technology, previously tested in balloon flights (see section \ref[balloon_flights]), was transitioned to, enabling vehicle location sharing on a unified map on the webpage. This service, augmented with Jitsi\urlnote{https://meet.jit.si/} for voice communication, offered a more functional solution. However, challenges persisted, in the form of poor audio quality from laptop microphones and unresolved integration of vehicle hands-free systems with measuring notebooks connected to the internet via onboard modems, because the connection interfered with connecting the driver's regular smartphone used for navigation.

\secc Data capturing procedure

Before each expedition, selecting the right time and location for observation was essential for the experiment. This process involved a decision-making sequence that began with long-term weather forecasts to identify potential stormy days, followed by ECMWF model results tens of hours in advance. On the measurement day, satellite images revealing cumulonimbus cloud formation and, subsequently, radar data were used to pinpoint the precise location suitable for measurements. 
Communication between cars, aside from coordinating measurements and positioning to ensure a non-linear arrangement of cars, is aimed at maintaining a strategic positioning relative to the highway for rapid relocation in response to storm development. Ensuring geometrically advantageous positioning of the vehicles relative to storm clouds proved challenging due to the linear nature of highways. Sometimes, opposing exits allowed for symmetrical positioning around the highway without straying too far. In other cases, exits before, within, and after highway curves were utilized. 
Quick communication was also essential for confirming data records, as the lengthy recording time necessitated verification that all vehicles recorded data simultaneously (otherwise the data recording should be omitted), a task managed manually through verbal communication, highlighting the need for automation in future stationary measurement station setups.

\secc Ionizing radiation measurements

Figure \ref[figur_radon] shows an example of ionizing radiation measurement using the gamma spectrometer AIRDOS-C. In the upper graph, a storm approaching a parked measuring car is displayed using data from the Blitzortung.org network. The vertical red lines mark the times when lightning was registered by the STP antenna. The lower graph shows the number of particles registered by the ionizing radiation detector every 15 seconds. In this particular example, the lightning activity ceased just after 20:30. The graph shows an increase in ionizing radiation flux by approximately 30 \%. This increase is related to a radon progenies washout from the atmosphere caused by rain that started at 19:45.

\medskip
\clabel[figur_radon]{Ionizing radiation flux compared to registered lightning}
\picw=15cm \cinspic ./img/figur_radon.png
\caption/f  Measuring ionizing radiation using a car compared to lightning registered by antennas. The beginning of measurements using  AIRDOS-C and STP antennas corresponds to switching on the devices after parking the car on a spot at 19:15. The measurement ended at 21:25 by switching off the devices and leaving the observation spot.  
\medskip

Figure \ref[fig_gamma] shows a detail of ionizing radiation measurement at the time when the storm was closest to the car. Individual particles of ionizing radiation registered versus detected lightning are shown. Only particles that have passed on energies higher than 2.4 MeV to the detector are displayed.


\medskip
\clabel[fig_gamma]{Individual registered particles of ionizing radiation of gamma spectrometer}
\picw=15cm \cinspic ./img/fig_gamma.png
\caption/f   Lightning detected by Blitzortung.org and triggers from STP antenna plotted together with individual registered particles of ionizing radiation above channel 30 of gamma spectrometer.  
\medskip

\secc Visual Camera measurements

Because lightning discharges often happen inside clouds and lightning channels are not directly visible, the videos of recorded lightning are converted to luminosity over time. The integral values of illumination are calculated from the video recordings of high-speed cameras using a script. The calculation is carried out over the entire image area by summing up the values of all pixels on each frame. These results are referred to as luminosity curves. The example can be seen on the graphs in Figure \ref[luminosity_curves], depicting the course of light flux over time.

To maintain clarity, luminosity curves are added to videos, which are converted from BW recordings captured by cameras to false colors to make details of lightning (that have high luminosity dynamics) visible.  

The following enclosed video [1627302288.9546976.mp4] shows lightning together with leaders and recoil leaders. During the observation, the camera is pointed towards the zenith and its lens allows capture of the entire sky from horizon to horizon. For the illustration of the camera view a plain camera frame is shown in Fig.  \ref[figur_horizon]. During the thunderstorm recording, the camera shutter time is decreased to capture lightning that has high brightness dynamics. As a result, lightning is not visible on the camera together with details of clouds and surrounding terrain. The processed video has the position of the horizon marked with a green circle with an inscribed designation of the cardinal directions. In the upper part of the video, there is a white graph showing a luminosity curve with a green pointer marking the current position of the displayed frame on the time axis. On the right, the used color range is visible, corresponding to the brightness recorded by each camera pixel with a depth of 8 bits. The bottom line shows the number of the current frame over the total number of frames recorded. The next is Sg=1/1 (information on which part of the camera’s internal memory was used for recording) and the time of the current frame ‘T=’ in seconds relative to trigger time.

\medskip
\clabel[figur_horizon]{A still image example from the all-sky camera}
\picw=15cm \cinspic ./img/figur_horizon.png
\caption/f   A still image example from the all-sky camera showing horizon and fish-eye lens distortion.  
\medskip

The resulting luminosity curves contain morphological signatures, such as sharp peaks or slow changes in luminosity. Thanks to this, video recordings with similar luminosity curve progressions can be compared, and from several recordings, those parts can be chosen that have similar luminosity curves and a visible lightning channel (not covered by clouds). Such a process helps to reveal what lightning looks like.

\medskip
\clabel[luminosity_curves]{An example of luminosity curves}
\picw=15cm \cinspic ./img/luminosity_curve_examples.png
\caption/f   An example of luminosity curves for a few lightning events.  
\medskip

\secc Radio signal measurements

The advantage of using magnetic loop antennas to measure lightning is that observations are not disrupted by optically opaque clouds or daylight scattered in the atmosphere during daytime storms. However, when measuring the magnetic component of the electric field with a coil, only changes over time, specifically changes in the current flowing through the lightning channel, can be detected. A constant current in the lightning channel cannot be detected by a magnetic loop antenna. Figure \ref[figur_capture] shows an example of lightning recorded simultaneously by an STP antenna and a camera. The camera recording clearly shows a change in luminosity, with the sky being brighter before the lightning than after it occurred. Slow changes in brightness, indicating slow changes in current, are not visible in the antenna recording. Instead, clusters of fast pulses are observed, suggesting details about the development of lightning channels that are not visible in the camera recordings, likely because they are obscured by light from a constant current in the main channel. Pulses visible in both camera and antenna recordings are recoil leaders, and the most prominent short impulse is the CG return stroke.

\medskip
\clabel[figur_capture]{Example of VLF signal with the comparison to camera data}
\picw=15cm \cinspic ./img/STP_camera_data.png
\caption/f   Example of data from STP antenna with the comparison to camera data.  
\medskip

\secc Correlation of measured lightning with lightning detection network

The detection of lightning using the STP antenna has been compared with its detection using the Blitzortung.org network. It's important to note that not every lightning strike detected by the STP antenna is also detected by the Blitzortung.org network. Figure \ref[figur_blitz2] shows that when the storm was closest to the car according to Blitzortung.org, the STP antenna detected lightning at different times or at distances of more than 120 km. Conversely, as shown in Figure \ref[figur_blitz1], when the storm was located tens of kilometers away from the observation site according to Blitzortung.org, there is perfect conformity with the data from the STP antenna. In both figures, an interval of ±1 second is marked around the vertical lines corresponding to the detection times.

Lightning at 18:24:48 was detected by a high-speed camera, see video 
[1627302288.9546976.mp4.]. According to the video, one of the lightning channels occurred almost directly above the measurement car, but the nearest lightning detected by Blitzortung.org was at least 70 km away. The positioning accuracy in the case of the Blitzortung.org network is in the order of kilometers. Blitzortung.org detected discharges at a distance of 70 to 80 km simultaneously. This allows the deduction that this lightning was more than 80 km long or that a synchronous discharge occurred 80 km away.


\medskip
\clabel[figur_blitz2]{Correlation with less than 20 km lightning detections}
\picw=15cm \cinspic ./img/figur_blitz2.png
\caption/f   Correlation with “nearby” (less than 20 km) lightning detection.  
\medskip


\medskip
\clabel[figur_blitz1]{Correlation with more than 20 km lightning detections}
\picw=15cm \cinspic ./img/figur_blitz1.png
\caption/f   Correlation with “distant” (more than 40 km) lightning detection.  
\medskip


\secc Electric field measurements

Figures \ref[figur_efmdata] and \ref[figur_efmdatadetail] compare vertical electric field measurements at the car's measuring site with camera recordings. The Kleinwächter EFM 115 on the car is not grounded, so it measures the gradient of electric potential between the sensor and the car body. As shown in Figure \ref[figur_efmdata], electric discharges occur during times of significant electric field changes. However, as detailed in Figure \ref[figur_efmdatadetail], changes in the electric field do not directly correspond to individual discharges recorded by the camera. It's important to note that the EFM measures the vertical component of the electric field, which integrates over a large area of the cloud and is also affected by the presence of the car.

\medskip
\clabel[figur_efmdata]{Example of EFM measurement}
\picw=15cm \cinspic ./img/figur_efmdata.png
\caption/f   Example of measurement with EFM compared with records of lightning captured on the camera. 
\medskip


\medskip
\clabel[figur_efmdatadetail]{Comparison of EFM data with camera data in detail}
\picw=15cm \cinspic ./img/figur_efmdatadetail.png
\caption/f   Detail of EFM data with a comparison with camera data.  
\medskip


