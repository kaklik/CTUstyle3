\input ctustyle3
\worktype [D/EN]
\faculty {F3}
\department {Department of Radioelectronics}
\title {Radio Detection of Electromagnetic Phenomena in the Atmosphere}
\subtitle {Integrating Advanced Instrumentation and UAVs for Enhanced Atmospheric Research}
\titleCZ {Rádiová detekce elektromagnetických jevů v atmosféře}
\subtitleCZ {Implementace pokročilé instrumentace a bezpilotních prostředků pro atmosférický výzkum}
\author {Jakub Kákona}
\date {Jun 2024}
\abstractEN {This dissertation explores the use of radio detection for studying electromagnetic phenomena in the atmosphere, focusing on the development and integration of advanced radio receivers,  meteorological instruments, mobile measurement platforms,  telemetry systems, and unmanned aerial vehicles (UAVs). It presents a comprehensive approach to capturing and analyzing atmospheric data, offering insights into the complex dynamics of storms and electromagnetic events. Through a series of experiments, including atmospheric flights and ground-based measurements, the research demonstrates the potential of novel technologies to enhance the understanding of atmospheric processes, contributing significantly to the field of atmospheric sciences.}
\abstractCZ {Tato disertační práce se zaměřuje na využití rádiové detekce pro zkoumání elektromagnetických jevů v atmosféře. Hlavní pozornost je věnována integraci nových meteorologických přístrojů, radiových přijímačů, telemetrických systémů a bezpilotních prostředků (UAV) do výzkumu atmosférických jevů. Práce přináší detailní analýzu atmosférických dat získaných prostřednictvím leteckých a pozemních měření. Jejich přímé využití je potom nasměrováno k hlubšímu pochopení dynamiky bouří a atmosférických elektromagnetických událostí. Popisované experimenty demonstrují, jak aplikace pokročilých měřících technologií  může zlepšit pochopení specifických atmosférických jevů.}
\declaration {      % Use main language here
   Prohlašuji, že jsem předloženou práci vypracoval
   samostatně a že jsem uvedl veškeré použité informační zdroje v~souladu
   s~Metodickým pokynem o~dodržování etických principů při přípravě
   vysokoškolských závěrečných prací.

   V Praze dne 26. 6. 2024 % !!! Attention, you have to change this item.
   \signature % makes dots
}


\draft

\makefront


I had decided to write the relatively wide introduction part, because I hope that form should explain the context required to understand some tradeoffs created during the entire work. It also allows a reader to capture the broad spectrum of activities which without the explanation could seem unrelated. 
Therefore at the beginning of my research I had worked on the Bolidozor network. The Bolidozor is a type of forward scattering radio meteor detection network, which uses the signal from french military radar GRAVES (https://spp.fas.org/military/program/track/graves.pdf) non cooperatively.  That means the signal transmitted by the GRAVES is scattered by meteor trails in the atmosphere and is really easy to receive the reflections on broad parts of Europe even with using a simple ground plane antenna.  
There are a few motives to deal with that. The first, the most obvious one, the radar observation of  meteors are weather and daytime independent in contrast to yet widely known visual based observation using the cameras. That radio observation should fix that limitation where visual based observation networks have impaired sensitivity for a significant portion of the time. Another motivation is the possibility to enhance precision of velocity estimation of the meteoroid, because there is a significant effect of doppler transition observed on the reflected signal. 
The existence of a doppler shifted “head echo” on meteor reflection was the core handle, because I had planned to use it to estimate the meteor trajectory from signals received by multiple stations.  That seems to be feasible, because there were successful attempts (M. A. Vallejo, ea4eoz, 2016 et. al.) to calculate a meteor vector in the atmosphere based on these doppler shifted signals. 

\rfc{Tady musím doplnit obrázek obrázek detekce meteorů z více stanic}

The trouble begins with the fact that there is no easy way to verify that the calculated trajectory is correct or incorrect. 
The one issue roots in the situation that radio signals received by Bolidozor network have a detected meteor every few seconds which complicates clear assignment of the specific visual observation to the calculated trajectory. Especially in cases where a digital video camera occasionally has a few seconds latency or inaccuracy. The second issue is caused by the situation that  GRAVES radar guaranteedly lightens only a relatively small fraction of the atmosphere, but there are also side radiation lobes. The primary enlightened part is located above south europe where there were little video detection networks at that time. 
The GRAVES radar also has a side transmission from its antenna, but these transmissions are not stable and also there is not exactly known enlightened area. 

\rfc{Tady musím doplnit obrazek odrazů od starlinku}

That results in a very few meteor events, which could be used for trajectory verification by using the local video based meteor detection networks here in the Czech Republic. 
To resolve this problem I had decided to switch from GRAVES radar transmission to the local transmitters which are more suitable for local meteor detection.  I had selected the VOR beacons for airplane navigation. These beacons have definitely reduced transmission power compared to GRAVES, but according to the numerical model I constructed the meteor radar based on that transmitters could be feasible with the use of state-of-the-art radio components.


\rfc{Tady musím doplnit obrázek}
Figure: Reflection from airplanes clearly visible at the received signal from the VOR transmitter (Prague OKL). The doppler shift curves are related to the airplane trajectories. 

Unfortunately that new approach requires a complete redesign of the signal processing and construction of the new receiver. That receiver should be capable of reception of multiple VOR transmitters at once, because the frequencies of VOR transmitters are allocated in such a way that neighboring transmitters have significantly different frequencies to enhance the airplane navigation safety and reliability. That results in the requirement of processing the 10MHz of signal bandwidth instead of the previous 192 kHz including a wide dynamic range of signal input, because this bandwidth will be likely affected by the strong nearby signals like reflection from the airplanes visible in fig \ref[VOR_signals].  
These requirements on the receiver redesign were way beyond the initial project funding available; it is also beyond my alone manpower. Therefore I had steppily realized that the newly developed instruments needed to have commercial applications to avoid reliance on unreliable and discontinuous systems of scientific founding. As a result I had to search for possible commercial applications that required the new instruments. That also explains why the majority of the newly developed instruments described in section \ref[Proposed instrumentation] are currently commercially available. For the case of the new radio receiver I found following areas of possible applications: 

\begitems
 \item - Meteor trail detection and localisation
 \item - LEO satellites down-link ground station
 \item - Atmospheric electrical and ionization events observation
\enditems

Luckily there arise an opportunity to cooperate at CRREAT project which main aim was study of high-energy atmospheric events, where electromagnetic events in atmosphere (electrometeors) observations fits well and at the same time are vital requirement. In the frame of that project I had designed the new receiver (described in chapter \ref[UHF signal receiver]) concept with all mentioned applications in mind. That allows that construction of the receiver could be implemented with significant assistance of other members of the CRREAT team or external collaborators and with the use of CRREAT funding.
But at the same time, there is a threat that the observation of lightning has the similar issues as the Bolidozor’s radio meteor observation, because computing an location of lightning occurrence is definitely possible, but there is non trivial task to verify that the result is relevant. 
The requirement to build the ability to verify the calculated results, branched out in the broad range of different work packages, which needs to be solved to gather relevant information about lightning or more generally atmospheric electricity from multiple sources. I describe ground based, airborne and remote sensing instruments in detail in the chapter \ref[Proposed instrumentation]. But for the overview of the tasks, firstly the lightning should be detected, the antenna should be calibrated, and the electric field needs to be measured at the same time. The lightning also needs to be simultaneously captured on high-speed cameras for geometric triangulation etc. 
That is why I need to design and operate the multiple measurement systems carried by car on ground and also in airborne vehicles presented and used in the following thesis. 




\chap Introduction

\sec Scope of the thesis

The objectives of this dissertation are to address and clarify several unresolved phenomena related to thunderstorms, particularly the initiation and development of lightning. This research aims to develop and utilize new tools for the detection and scientific observation of lightning events within thunderstorms. The focus will be on understanding the relationship between ionizing radiation and lightning, as well as developing innovative mobile instruments required for detailed storm observation. The specific tasks to be undertaken are as follows:

\begitems
\item - {\bf Development and Implementation of Lightning Detection Apparatus}: Mobile experimental apparatus capable of radio detection and localization of lightning events and ionizing radiation, considering the expected spatial scales, will be designed and deployed. Data will be collected using radio detectors, high-speed cameras, and ground-based electric field measurements.
\item - {\bf Analysis of Spatial and Temporal Characteristics of Lightning Discharges}: The propagation of lightning, emitted radio signals, and associated electric fields will be studied. Methods for visualizing lightning discharges that avoid misinterpretation as ground strikes and more accurately reflect the physical principles of lightning development and radio signal emissions will be developed.
\item - {\bf Enhancement of Measurement Techniques with in-situ atmospheric monitoring}: New instruments such as stratospheric balloons and unmanned aerial vehicles (UAVs) (especially unmanned autogyro) for in-situ measurements of electric fields and ionizing radiation will be developed. The new electric field mill and a semiconductor-based ionizing radiation detector will be integrated into the UAV.
\enditems

The results of this research are expected to provide a deeper understanding of lightning initiation and propagation. At the same time it offers improved detection and observation techniques that can inform future studies, about relation to ionizing radiation with direct practical applications in meteorology and atmospheric physics.

\input introduction.tex 

\chap Lightning triggered recording 

\input triggered_recording.tex 

\chap Proposed Instrumentation

\input instrumentation.tex 

\chap Subsequent Experiments

\input experiments.tex 

\chap Summary of results

\input results.tex 

\bibchap
\usebib/c (simple) references

\bye
