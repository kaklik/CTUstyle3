Chapter is based on the published article: https://amt.copernicus.org/articles/16/547/2023/
The discussed video content is also included in article assets. The following text highlights the main findings and results achieved from the date of publication.

\sec Lightning duration

The duration of the lightning event was determined by identifying local maxima of illumination on the camera chip over time from peaks that had a prominence higher than twice the noise level. The first and last maxima were used to identify the first and last frame, and the lightning duration was calculated from the framerate. The first ten frames of the camera recording were omitted to avoid potential artifacts from video data compression. The noise amplitude was established from the first 100 frames (from frame 10 to frame 110). Examples of duration determination for some lightning events are shown in Figure \ref[Camera_lightning_duration]. The histogram of lightning event durations, shown in Figure \ref[duration_histogram], includes 107 lightning events measured during 12 thunderstorms.

\medskip
\label[Camera_lightning_duration]
\picw=15cm \cinspic ./img/camera_lightning_duration.png
\caption/f   Examples of detection of lightning events duration. Red crosses are detected peaks in signals generated by lightning processes. The determined length of lightning duration is highlighted. Duration of the camera frame is 620 µs.  
\medskip

To calculate the total length of a lightning event from a radio signal (see Fig. \ref[UHF_lightning_duration]), a similar methodology was used, where a signal greater than 4σ variance, calculated from a moving window of 10 ms, marks the beginning of the lightning discharge. This algorithm is applied symmetrically from both ends of the recorded signal. For detecting the lightning start, the process begins from the first sample of the signal, and for detecting the lightning end, the process begins from the last sample of the recorded signal.


\medskip
\label[UHF_lightning_duration]
\picw=15cm \cinspic ./img/QFH_lightning_duration.png
\caption/f    Example of lightning duration measurement based on radio signal captured by the UHF receiver. The time duration of lightning is marked by the green overlay.  
\medskip


\medskip
\label[duration_histogram]
\picw=15cm \cinspic ./img/Duration_histogram.png
\caption/f   Histogram of the observed duration of lightning events. The histogram is based on data captured by high-speed cameras.  
\medskip

The measured durations are usually in the order of hundreds of milliseconds, only exceptionally there are shorter lightning events. The median duration of lightning events is 524 ms. This result is significantly higher than was mentioned in previous studies 200-300 ms \cite[rakov_uman_2003]  and 350 ms  \cite[López_Pineda_Montanyà_Velde_Fabró_Romero_2017]. 

The processing of lightning recordings differs for day and night storm observations as they require different settings of high-speed cameras, in particular, the setting of exposition length and analog gain varies. This difference could possibly affect the exact measurements of the total length of a discharge, resulting however only in a shortening of the estimated time duration. Because of sunlight scattered in the atmosphere during the daytime, the weak discharges might have been omitted. Therefore the extracted time durations of lightning events could be possibly underestimated. 
From the subsequent analysis of the lightning duration, two thunderstorms were excluded where it was not possible to distinguish individual lightning events, meaning there was no delay of at least 500 ms between the individual detected discharges. Lightning is considered to be an event where the time between the individual discharges does not exceed 100 ms.

\sec Lightning duration

Data from the cameras reveal similar phases of lightning development. Examples of video recordings with individual phases clearly visible and not obscured by clouds are provided.
Video [1627302745.846055.mp4] captures (from T=+0.27 s (1:11) to T=+0.40 s (1:22)) a positive side of the leader. When the current flowing through the leader begins to weaken, recoil leaders start to appear at its end (T=+0.38 s (1:21)). The term recoil leader was taken from literature \ref[MAZUR2013763]. Based on the described observations, however, it is not confirmed that all the visible recoil leaders reuse an exactly already established ionized channel.

Video [2021-08-15-20-07-35.912167-lightning.mp4] captures invisible positive leader, only blurred recoil leaders are visible. From T=+0.07 s (1:36) a negative side of the leader is visible. Negative leader branches abundantly, its propagation is faster than positive leader and contains hot luminous ends. The second visible negative leader starts at T=+0.25 s (1:51). The negative leaders do not generate recoil leaders.

Based on the experience from recordings where some parts of lightning are clearly visible, the cloud obscured lightning events can be characterized. There is an example of the luminosity curve compared to beside the and STP antenna signal data shown in Figure \ref[figur_capture]. At the very beginning, the lightning usually starts with a faith peak with a fast-rising edge (at time 0.2 s). Then there is an optically dark phase with low luminosity. This dark period is not really quiet in the radio signal. It is a phase of developing a leader. As the leader in the cloud connects to more and more charged regions it becomes brighter, sometimes slowly and sometimes abruptly. During and after a decrease in the current in the positive leader the recoil leaders are appearing (after time 0.4 s). Some recoil leaders have high luminosity when they connect to the main channel. In the radio spectrum, simultaneous peaks are observed that correspond to short, intense brightenings. In some cases the lightning results in CG discharge, which is accompanied by high-intensity flash (at time 0.5 s). However, this phenomenon does not occur very often. Using the described equipment, CG return strokes were registered in less than 10\% of cases.
The described findings have direct implications, for example, on the way lightning discharges events should be displayed and localized. Given the typical dimensions and time scales on which discharges occur, it follows that even receivers operating on the VLF principle can provide meaningful descriptions of lightning discharge structures. At the same time, for the same reason, it is misleading to display lightning discharges on maps as points, which are intuitively interpreted as CG lightning strikes hitting the ground at approximately the position of the map mark. In reality, most of those points actually correspond to CC lightning discharges, given the actual incidence of their occurrence over CG lightning.
A better representation that considers this fact is attempted in Fig. [BLESK], which shows an calculated approximate visualization of the lightning structure. The representation is based on VLF data obtained from only three measuring stations placed on cars, so it contains many simplifications. For example, the lightning structure is depicted in a 2D plane positioned at a fixed height above the ground. Additionally, it is important to note that lightning discharges do not radiate at branching points, but primarily at the locations of two-thirds recoil leaders. The drawn connections do not reflect this fact, as the depiction is currently an illustrative algorithm that, despite its limitations, is a closer description of the physical principle of lightning discharges than the widely used depiction of points in a map.

\medskip
\label[VLF_lightning_map]
\picw=15cm \cinspic ./img/VLF_lightning_map.png
\caption/f   An illustrative visualization of the structure of lightning discharges based on VLF data obtained from three measuring stations mounted on vehicles. The representation depicts the lightning structure in a 2D plane at a fixed height above the ground, capturing the approximate spatial distribution and progression of the lightning channels. The depiction emphasizes the complexity of the lightning and provides a closer representation of the physical nature of lightning discharges.  
\medskip

The algorithm used for computing and visualizing lightning uses VLF recordings obtained simultaneously from three measurement vehicles by the described equipment. These recordings are aligned based on absolute time marks. The data fragments are normalized and their offsets removed. The prepared recordings are then divided into a predetermined number of segments. Segment length is not constant but is chosen so that each contains a similar amount of energy. The energy level and segmentation are algorithmically determined based on the total energy of the weakest signal. This ensures that each segment divided from the lightning signal fragment has enough significant morphological features that enable calculation of signal shifts. TDoA values are then calculated by cross-correlating pairs of signals across the three stations.
The localization algorithm utilizes segments with at least two TDoA values that match physical constraints (e.g., time shifts cannot exceed the physical distance between stations). The navigation solution is then obtained by numerically solving the TDoA localization problem within an area situated above the plane where the stations are placed, at the expected cloud base altitude.
The fragment signal processing method is selected due to the aforementioned results. Additionally, the use of cross-correlation over signal segments provides significant advantages when one of the signals (in this case, CAR1) is heavily interfered by uncorrelated noise absent from other stations. That is a feature which cannot be applied on conventional lightning detection networks, where only individually recorded pulses are processed. 
The visualization is achieved through a graph algorithm that progressively adds points to the graph in a way that satisfies the geometric conditions related to lightning propagation, like utilizing adjacent lightning channels. This approach effectively eliminates the need for clustering algorithms commonly used \cite[Sibolla2021] to determine the lightning strike area, as the connection between individual parts of lightning is not lost in the process. The described algorithm could be significantly improved by implementing an algorithmic estimation of the optimal number of segments based on pre-analysis on the lightning development in the signal fragment, that could be probably done by some form of identification of lightning phases in the signal. In other words, by creation of a lightning signal model which should be then matched on the real signals from different stations.  


\sec Conclusions

During multiple summer storm seasons, data on more than 100 lightning events were collected using the described measuring apparatus and methodology, which includes multiple innovations to which I contributed, namely the instrument triggering method and new sensors. Initially, the apparatus was designed to detect and localize lightning events within a spatial range of a few kilometers. However, it was discovered that the spatial extent of these phenomena (based on lightning observations) is likely tens of kilometers, rendering the initially designed apparatus inadequate.
Observations made using the proposed apparatus (cameras, radio detectors) revealed that the data from all-sky high-speed cameras and magnetic or electromagnetic antennas provide comparable results in terms of lightning duration measurement. The median duration of lightning measured was significantly longer than previously stated in the literature. Furthermore, stationary ground-based electric field measurements (using field mills) did not correlate well with the manifestation of lightning discharges and provided limited information about the development of lightning, indicating that such measurements should be performed in situ using UAVs.
The research faced significant challenges due to the primarily horizontal propagation of lightning. This suggested that the electric field and resultant RREA would also radiate horizontally, rather than vertically. This finding was supported by the failure to detect ionizing radiation from below storm clouds, while stationary detectors on elevated locations detected it even in the absence of nearby lightning.
Additionally, optical and radio observations of lightning discharges demonstrated that the initiation of lightning discharges could not be confined to a short time interval around the return stroke but developed over several seconds. The return stroke, often considered a key phase, appeared to be just one of several possible phases and occasionally did not occur at all.
The situation necessitated the expansion of measuring instruments to include a broader range of devices. These included stationary detectors on elevated sites and in-situ measurements (of electric fields and ionizing radiation) using UAVs. Initial tests with stratospheric balloons showed promise, leading to efforts to use the autogyro for controlled in-situ measurements, as balloons are impractical for stormy conditions. This required the design of a new electric field mill and a semiconductor diode-based ionizing radiation detector suitable for UAVs.
The extended experiments revealed that the initiation of lightning discharges likely occurs far from where the visible discharge is observed. Despite capturing one video potentially showing the initiation phase, it has not been directly used due to the complexities in interpreting it due to missing contextual data from other instruments. Consequently, simple correlation between ionizing radiation detection and lightning discharge is insufficient to determine causality, as the required experiment is significantly more complex and demands a much larger instrumentation coverage area or an invention of method, which allows to focus solely on the initiation phase of lightning.
The current implementation of radio detection networks for lightning has proven inadequate. These networks process lightning as clusters of impulses, losing significant information in the process, and present data as "crosses on a map," often misinterpreting them as return strokes, which were almost never actual ground strikes. The research suggests a better representation of lightning discharges, avoiding graphics that imply ground strikes and considering the physical development of lightning, especially the creation of VLF radio signals primarily emitted at segments where ionized channels reconnects (recoil leaders). Although this representation is still simplified and requires further work, it is a significant improvement over current implementations.

\sec Future Work

Future research needs to be  focused on in-situ measurements. UAVs, despite their limited success in the currently described work, appear to be ideal for this purpose. It is necessary to measure the electric field in space (ideally as a 3D vector), which an autogyro, with its rotating rotor, could facilitate. That type of in-situ measurements should also include ionizing radiation detection.
The deployment of these approaches in subsequent studies is expected to yield substantial results, possibly confirming direct detection of RREA and the directional structure of thunderstorm ionizing  radiation sources. That knowledge is essential to interpret the measured ionizing radiation and lightning event data in order to explain lightning initiation or eventually predict the lightning occurrence. 
The application of the TF-G2 autogyro in the described study has highlighted the potential and challenges of leveraging UAV technology for atmospheric research. The issues primarily stemmed from the software's maturity, which significantly impeded ability to conduct successful measurements during storm activity. This aspect of technology, while a limitation in the current study, offers a critical area for improvement and optimization in future work. Enhancing the stability and reliability of UAV related code will undoubtedly augment higher capability to collect high-fidelity atmospheric data from nearby or within storms, thereby enriching the understanding of thunderstorm dynamics, improving weather prediction models, and contributing to climate studies. Therefore continuous development and field testing of the unmanned autogyro and related technologies are required.



