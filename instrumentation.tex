To be able to record the different processes occurring during thunderstorms, the three cars were equipped with the specialized measuring equipment. The aim is to verify the hypotheses  considering the relation between thunderstorm and ionizing radiation stated in the \ref [Introduction] chapter, because these cars were able to move to locations with predicted storm activity and thus flexibly react to specific storm developments, and perform ground measurements directly at the lightning site.

In order to determine the necessary parameters of lightning activity (lightning events’ timestamps, lightning type, and location), the measuring cars, see Fig. \ref[CRREAT_CAR0] (CAR0 and CAR1), were equipped with a high-speed all-sky camera and radio receivers. The cars were used to transport and power the instruments in proximity to thunderstorms, and the car cabin also served as partial protection for the instrument operator.

The equipment of cars is not uniform as it is gradually improved and also the purpose of each car slightly differs.

\medskip
\label[CRREAT_CAR0]
\picw=15cm \cinspic ./img/CAR0_instrumentation.jpg
\caption/f Measuring CAR0 with instruments mounted on the roof platform. Other cars were equipped similarly. The differences of instrumentation are depicted in diagrams. 
\medskip

\sec High-speed all-sky Cameras
 
The two cars were equipped with high-speed all-sky cameras. Their purpose is to capture the lightning event evolution in parallel with recording from the other instruments, especially radio receivers. 

Chronos 1.4 camera  CR14-1.0-16M [chronos_datasheet] was mounted in a waterproof SolidBox 69200. The box is covered with a plexiglass dome with the manufacturer designation Duradom 200mm depicted in Fig high_speed_allsky_camera. 


